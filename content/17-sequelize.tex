\section{Sequelize}

Sequelize is one of the oldest, still popular, and actively supported ORM
frameworks for the JavaScript ecosystem, with development starting in 2010.
After over 12 years of development, it supports various databases, including
PostgreSQL, MySQL, and SQLite3 \cite{sequelizeDocs}. Its popularity is
attributed to its simple syntax and flexible usage. Sequelize supports read
replication, using round-robin scheduling for read queries and a write pool for
write or explicitly marked database calls. It also provides a command-line
interface for managing schema and data, offering migration and seeding options.

Sequelize can use either manually written migrations or automatically generated
ones based on Model definitions. The framework also supports programmatic
synchronization of the database state to reflect the Model definition using the
\texttt{sequelize.sync()} method. Unlike other ORMs discussed in this work,
Sequelize does not expose a query builder interface, only allowing for raw query
execution when integrated entity fetching is insufficient.

Full TypeScript support for the main project is still in progress
\cite{sequelizeTypescriptSupport}. However, a separate project called
\textit{sequelize-typescript} \cite{sequelizeTypescriptNpm} offers support for
TypeScript native definitions, using decorators for Model and property
specifications. One of the original features of Sequelize is its support for the
automatic creation and update of timestamp fields on database tables and the
option for "paranoid" tables \cite{sequelizeParanoid}. In this mode, calling a
deletion method sets the \texttt{deletedAt} attribute instead of removing the
record. The framework behaves like the deleted records do not exist unless
previously restored.

Both Sequelize and sequelize-typescript are provided under the MIT Licence and
actively supported. Due to its extensive user base, the framework has over 750
reported issues \cite{sequelizeGitHub}, with a minority marked as bugs and the
majority being feature requests. The central repository of the package has
27,729 stars on GitHub and the largest number of weekly downloads out of all
packages compared, at 1,505,485 \cite{sequelizeNpm}.

Sequelize has 16 dependencies, plus TypeScript-specific helpers required for
usage with the current TypeScript version. These dependencies primarily relate
to inflexion, code analysis for model dependency graphs, and various data types
supported by the framework, including GeoJSON and advanced Date types. While the
package offers extensive documentation, differences in Model definition when
using, TypeScript can be confusing. Documentation for additional operators and
decorators is lacking, as it is primarily contained in several Markdown files in
its repository.

\subsection*{Performance in benchmarks}

In performance benchmarks, the Sequelize ORM demonstrated competitive results,
closely matching its main rival, TypeORM. Sequelize generated efficient queries
for basic operations and performed within 10-50\% of the reference pgTyped
implementation. However, with more complex queries, Sequelize preferred using
LEFT OUTER JOINs when they were not necessarily based on the conceptual schema
of the database. Consequently, these test scenarios took 50-150\% of the
reference solution execution time.

While query execution remained relatively fast, the increased complexity due to
unnecessary LEFT JOINs may lead to performance issues that necessitate manual
optimization in some instances. As the LEFT JOINs are some of the more complex
operations databases perform, and their complexity grows rapidly with the amount
of data joined, this approach can quickly impact the application's overall
performance.

\begin{listing}
    \caption{Example of nested where condition in Sequelize - countCatsByColor test}
    \label{lst:SequelizeWhereType}
\begin{minted}{typescript}
return Cat.count({
// Nested objects are not supported for filtering
// The key is not typed either
  where: {
    '$catColor.colorHex.hex_code$': hexColor,
  },
  include: [{ model: CatColor, include: [ColorHex] }],
})
    \end{minted}
\end{listing}

Sequelize's filter types were limited, as even essential attribute matching for
search by attribute equality is not typed. This lack of typing support increases
the likelihood of developers making errors that may only become apparent during
runtime. Furthermore, Sequelize does not provide a dedicated API for WHERE
clauses on joined tables, requiring developers to rely on dot notation and
unchecked fields to define such filters.

While defining models is documented, the example code provided was found to
throw errors related to shadowing attribute names. Although the solution to this
issue can be quickly addressed using the TypeScript "declare" keyword, it is
essential to mention such features in the introductory documentation for models
to minimize confusion for developers.

Like most other ORMs, Sequelize does not support abstracting upsert operations
with increments or decrementation on updates. As a result, developers must
resort to implementing such operations using raw queries.