\section{Zapatos}
The Zapatos package, developed by George MacKerron, is designed to provide type
safety for database querying in TypeScript, specifically for PostgreSQL through
the \texttt{pg} driver. It generates a TypeScript schema of the database via
introspection, offering methods for basic CRUD operations that are instantly
typed and function as shortcuts for generic queries. Additionally, it features
tagged templates for writing arbitrary SQL \cite{zapatos}.

Unlike pgTyped, which analyses queries and generates types, Zapatos disallows
the inclusion of any value which was not specified in the manually types. This
approach can lead to issues if the type is not defined correctly initially,
potentially resulting in an unreliable type and is less explicit way to cast the
result. Despite this limitation, the package supports lateral joins, enabling
the return of nested objects, a feature typically not found in packages with
such low abstraction levels. However, as with other TypeScript-dependent
packages, Zapatos can generate compilation errors that are difficult to parse.

Released under the MIT license, the package uses ESModule dependencies and has
14,933 weekly downloads and 980 GitHub stars, ranking it third least downloaded
and starred. It is regularly updated, and none of the reported issues in the
repository significantly limit its use. The package has no runtime dependencies,
only development dependencies.

While the documentation quality is generally good, it is brief but covers most
of the necessary information for getting started, although lacking in detail.
Similarly, the annotations for types provide only brief information,
insufficient for guiding development on their own.

In summary, Zapatos is a package that provides type safety for database querying
with PostgreSQL, offering typed CRUD operations and tagged templates for SQL.
Although it takes a different approach to type generation than \texttt{pgTyped},
it supports lateral joins which result in nested objects which are natural way
of representing data in OOP. The package has concise but sufficient
documentation for most users \cite{zapatosDocs}. However, the package's
limitations include possible type inaccuracies due to developer error and
challenging compilation errors. 

\subsection*{Performance in benchmarks}

In the performance evaluation, the Zapatos package proved notable, as it was,
for example, among the few packages that permitted the utilization of distinct
objects for updates and inserts during \texttt{upsert} operations without
requiring a complete query. This package exhibited low latency in explicitly
written queries and shortcut functions, typically only surpassed by
\textit{pgTyped} or \textit{@database/pg}, which both employ a more rudimentary
representation of the SQL language.

However, a significant challenge was encountered during the Big Integer data
type test, as the identifier's value was altered due to its conversion to a
number before being returned. This issue was resolved by employing an explicit
SQL query instead of the shortcut function \texttt{db.selectExactlyOne}. This
unexpected result is particularly striking, given that the framework conducts
introspection over the database, correctly identifying the schema as
\texttt{int8}. Nevertheless, the shortcut function converts the column value
into a JavaScript number runtime type, which, according to the specification, is
represented by \texttt{C++} double (\texttt{float64}) or equivalent
\cite{MDNNumber}. Although both PostgreSQL and JavaScript use the same number of
bytes to represent the value, the PostgreSQL type can accommodate a
significantly more extensive range of integer values, as it utilizes decimal
precision \cite{PostgresNumeric}.

\begin{listing}
\caption{Zapatos number precision loss}

\begin{minted}{ts}
// Value incorrectly converted into a number
// Also see the specification of accessible 
// and selected values as template types for the sql
db
  .selectExactlyOne('cat', 
    { cat_name: name }, 
    { columns: ['id'] })
  .run(pgPool)
  .then(data => BigInt(data?.id ?? 0))

// Correct value returned as string
db.sql<
  schema.cat.SQL,
  schema.cat.Selectable[]
>`SELECT * FROM ${'cat'} 
  WHERE ${'cat_name'} = ${db.param(name)}`
  .run(pgPool)
  .then(rows => BigInt(rows[0].id ?? 0))
\end{minted}
\end{listing}

It was necessary to compose numerous queries using explicit SQL (as depicted in
the final comparison table), including transaction rollbacks. The framework only
performs this task autonomously if the transaction encounters an error (and
subsequently rethrows the error).

A unique and valuable feature of the Zapatos package is its support for lateral
joins, which the framework advocates. This approach generates nested objects
that align more naturally with the object-oriented programming paradigm.

Initially, due to limitations in the typings of Objection.js, the framework had
to be compiled with the \texttt{StrictNullChecks} option of TypeScript disabled. This
decision conflicted with the way Zapatos does its type checking. This option is
one of the most common problems with out-of-date type definitions, so Zapatos
will not be valid for projects that have to keep this option disabled due to
other dependencies.