\setsecnumdepth{part}
\chapter{Terminology Used}
\section{Object-Relational Mapping}

Object-relational mapping is a way to access relational data in an object-centred programming language. The primary purpose is manipulating data without switching concepts from object-oriented paradigms to the relational representation of data in which most databases operate. The scope of this translation layer can (as shown later in this work) vary. Different people define packages as ORMs while providing diverse levels of functionality. \par
At its base level, ORM provides an intermediary layer between applications' OOP model and database which is usually relational (but can be graph or document focused). The layer allows the developer to work with objects in the code, while the package translates it into a relational structure when saved to the database. These packages are often used on projects that are heavily connected to a database model, as ORMs are most beneficial when using a database is commonplace. When used only occasionally, it usually brings too expansive a setup to translate into gains in code readability and maintenance costs compared to executing premade SQL queries. \par
In addition to the basic functionality of translating between different styles of data representation, ORMs often include functionality such as connection pooling, support for read-only data replications, caching, or database migrations. When using such modules, developers can avoid writing boilerplate code that is typically required. \par


\section{SQL Query Builder}
SQL query builder is derived from its function to create SQL queries and OOP pattern, which it implements, called "builder". Object-oriented programming design patterns are reusable solutions commonly encountered during software development in OOP languages. These patterns propose interactions between objects and their internal structure. There is no single authority on how these patterns are defined, nor a comprehensive list of these patterns, as every author prioritises different patterns and functionalities. \par
The Builder pattern is one such pattern, providing API for the complex creation process of objects. This pattern is one of the 23 defined in "Design Patterns" by Erich Gamma, Richard Helm, Ralph Johnson, and John Vlissides, which has been highly influential in software engineering. Its purpose is to separate the creation of an object from its representation, allowing for the separation of parts that initially were parts of one construction method into several.\par
SQL Query is made up of several clauses, which each serve distinct functions. For example, the SELECT clause specifies which columns are to be retrieved, and the FROM clause specifies which table or tables the columns should be. Once the builder pattern is applied to the SQL query, each of these clauses (or even smaller fragments) can be created by calling the Builder's methods, creating a programmatical way to create SQL queries. The Builder also allows developers to abstract minor differences between different SQL implementations.\par
As the ability to create queries based on multiple criteria is one of the basic functionalities of ORMs, they are almost always built on some query builder. These can be available standalone or fully integrated into the ORM package. Often, query builders are sufficient for the purposes of database access in most applications, so they were included in the comparison. While they certainly lack feature sets and, compared to ORMs, query builders usually require SQL knowledge, they can be easier to set up and maintain while also being faster and allowing fine-tuned adjustments to a query. \par

\section{PostgreSQL}
One of the most popular database engines today, PostgreSQL, is an open-source object-relational database management system. Originally developed under the name Postgres (short for Post-Ingres) as a new generation after a successful relational database called Ingress, it was first released under this name in 1989. After several years of development at the University of California at Berkeley, the name was changed to focus on SQL compliance. The whole project moved to open-source, community-focused development in 1996. Currently, the project is maintained by The PostgreSQL Global Development Group, and releases and source code are provided under an open-source BSD-style licence free of charge.\par
PostgreSQL is, at the time of writing, one of the most popular SQL databases available, owing to its widespread adoption to reliability and high scalability while supporting most of the SQL standard and being fully ACID compliant. ACID is an acronym for Atomicity, Consistency, Isolation, and Durability, which are four fundamental tenets specifying properties for reliability and consistency of transactions (explained in TODO).\par
Some of the features that often make PostgreSQL stand out amongst other RDBMS are its support for many different and advanced datatypes out of the box, such as the ability to natively store JSON objects and arrays, XML data or geometric types. With extensibility being a significant focus, a lot of functionality can be installed or optionally enabled, further improving the reach and applicability.\par
There are extensive implementations of the API for many programming languages, including C, Python, Java, and JavaScript. PostgreSQL also offers extensive documentation of both its API and internal functionality, which supports its growth and popularity.\par

\section{Lazy loading}
Lazy loading is a technique for optimising data retrieval to increase application performance. It is a strategy consisting of only loading data when it is needed rather than all at once, therefore, reducing the initial load in exchange for the need to do additional loading later. Lazy loading is usually achieved by breaking down larger datasets into smaller ones and loading each one only when necessary. Such practice is commonplace in web development, as asset sizes (such as images or JavaScript) have only grown in the evolution of the web.\par
Some of the most common implementations of Lazy Loading come in the way of only loading low-quality images unless the user is focused on them or splitting code into multiple files, which are fetched when necessary, providing quick first page-load time at the cost of adding additional requests.\par
When talking about ORMs and database access, lazy loading usually takes place as replacing data retrieval from an object with a call to retrieve the data from the database. In other words, the data of the database object do not need to be loaded when the object representation or its part is created in the programme. This technique is often used for loading relations so that only one table needs to be queried to create the essential representation of the object while skipping the need for additional fetches or joins only to be invoked once needed.\par

\section{Eager loading}
Eager loading is the programming practice of loading all the required data at once, optimising the number of requests that must be made to retrieve everything. This is done with the expectation that one significant request will minimise the amount of additional data that would need to be sent between the two parties. This can lead to faster load times and improved application performance.\par
It is usually achieved by sending a singular request and caching the data in memory, even though it might only be needed later. There are obvious downsides to this, such as higher memory usage or often loading more data than is necessary. The concept of eager loading is antithetical to the lazy loading approach, and that is on purpose. Each approach prefers a different focus, and thus each is fit for different usage; lazy loading is practical when a first look or first results matter the most, and eager loading is when the focus is on one large result, which would be slowed down by too many small requests, which would need to be made for the total result.\par
In the context of object-relational mapping frameworks, we are most likely to encounter eager loading when fetching related entities. This way, when there is an expectation for data about the currently approached entity, the ORM can optimise the query so that the data are already loaded in memory when it is requested later.\par

\section{Circular dependence}
A common problem in software development, circular dependency occurs when two or more parts of code depend on each other, making it impossible to resolve their dependence onto a dependency graph. Such a graph must conform to limits set out for tree graphs and, therefore, cannot contain a loop. Due to the way how modules are loaded in Node.JS, such a problem would lead to a deadlock and is therefore resolved by trying to run the modules in a specific order. However, such an approach is only sometimes feasible, so other solutions must be used. The issue of circular dependency is also present in the compilation because, while TypeScript does allow asynchronous references of types between files using "import type ", if we need to import not only the type but also the value, TypeScript will not be able to resolve the type, and the compilation will fail. There are many solutions to this problem, the most common being dependency injection or lazy loading.\par
In ORMs and database representation in OOP languages, this problem is generally connected to the bidirectional nature of relations, as its explicit representation will inevitably create circular dependency. Therefore, there needs to be a functionality built in that allows users to define bidirectional relations without sacrificing type safety or encountering a deadlock with importing modules. \par

\section{Database transaction}
Transaction isolation is a concept used in database management to represent a unit of work. The transaction is typically a series of one or more database operations that are supposed to be completed on the all-or-nothing principle. In addition to performing database queries atomically, the transaction also needs to provide additional functionality, such as coordination of reads and handling operations in a reliable and recoverable manner. \par
As database transactions are some of the basic functionalities of modern RDBMS, their handling is essential when considering the ORM framework. Often an operation can only be performed when the previous one succeeded or has to be made strictly in order without another operation having access to the data in between. This can be achieved only through the database transaction, and support for them is necessary for many use cases.\par

\section{Database connection pool}
A database connection pool is a component that collects and manages several database connections and allocates them to individual requests to the database. It works by creating either a fixed number of connections at the beginning or scaling up the number of connections based on usage. In this way, querying the database does not have to wait for the connection to be established, and the request can be routed through the database connection pool to the currently unused connection. Additionally, due to having multiple connections, multithreaded and asynchronous applications can coordinate connections to the database. Single connection applications can be stalled while waiting for a single otherwise non-blocking request, while others could be served by the database. Such connections must be coordinated with transaction management, as the transaction is inherently connected with the connection that spawned it. \par

\section{Read replica}
A read replica is a special kind of database instance, a read-only instance of the database presenting additional query points for the applications accessing the database without having to resolve consistency between instances. With usual databases supporting multiple instances, concurrent writes to alternative machines could produce an inconsistent state in the database. With a read-only replica, consistency is not threatened; the only negative is the possibility that the connections will receive a state that is delayed when the replica is not synced to the latest consistent state of the primary instance.\par
Creation and usage of read replicas can significantly speed up database performance as queries are no longer constrained by single hardware, which usually bottlenecks query speed. Duplicating the data over two instances can double disk read speeds; if different physical devices are used, slow sequential scans over data can run independently and finish faster.\par

\section{JavaScript}
A high-level dynamically typed programming language developed in the mid-1990s at Netscape Communications Corporation to add dynamic content to web pages. Initially called Mocha, it was later renamed multiple times to finally settle on JavaScript to use the (at the time very high) popularity of Java.\par
Before JavaScript, websites were almost always purely static documents that were displayed in web browsers (such as Netscape at the time or Google Chrome or Firefox currently). The logic for any web application had to be handled purely on the server side. With the introduction of JavaScript, web pages were able to be more interactive and dynamic. While initially designed to be used when writing HTML documents and executed by web browsers, it outgrew its client-side roots and conquered large parts of the server-side development and even mobile app and desktop application environments. The advantage of JavaScript is that it can be a completely full-stack language that provides exact parity of logic between client and server and allows for significant code portability.\par
Until the last few years, JavaScript had an exclusive reign over interactive web content, which made it one of the most used programming languages in the world. With multiple deficiencies known and unfixable without massive problems with incompatibility, multiple additions which build atop JavaScript and even whole languages which compile into JavaScript were developed. Some complied languages are, for example, CoffeeScript, Dart or TypeScript. These languages exist to provide additional features and functionality that are not easily or at all possible in pure JavaScript.\par

\section{ECMAScript}
Soon in the usage of JavaScript for web pages, it became apparent that establishing standards would be a necessary step for compatibility between implementations in different web browsers. Following this consensus, Ecma (originally an acronym for European Computer Manufacturers until 1994) International standards association meeting was held, and the first edition of the document specifying the new standard specification was adopted in June 1997.\par
The document, coded under the name ECMA-262, is a comprehensive document that has gone over several versions over the years and specifies the syntax, semantics, and behaviour of the language. There is also an extensive description of data types, operators, flow control structures, built-in objects, and API.\par
ECMAScript is currently used primarily for client-side scripting, with primary implementations being those used in web browsers, such as SpiderMonkey (Firefox), V8 (Google Chrome, Opera) and JavaScriptCore (Safari). Increasingly with new revisions of the standard, even server-side applications and services have started migrating to ECMAScript from other standards (primarily CommonJS), but many constructs are not directly compatible or translatable.\par

\section{CommonJS}
One of the alternative specifications which reflected missing functionality in ECMAScript was CommonJS. Created to establish conventions on modularisation for JavaScript outside the web browser, it has also standardised several APIs and internal features. \par
Started in 2009 by an engineer at Mozilla, the project was initially called ServerJS, with its flagship feature being the synchronous loading of modules. This means that once a module is imported, its exported components are immediately available to be used. This simplifies working with modules and was necessary for the expansion of JS code into server-side development and is used widely today. \par
Since its conception, gripes with the ECMAScript specifications were largely fixed with further iterations, making it also usable in server-side development. Popular packages, including those exclusively used in server development, have migrated their codebase to ECMAScript.\par

\section{TypeScript}
A statically typed language built on top of the JavaScript foundation, TypeScript was developed by Microsoft Corporation with the focus on allowing developers to catch errors at compile time before the problem is encountered during runtime, which usually requires extensive testing. TypeScript code is written in enhanced syntax and then compiled into regular JavaScript, with several standards supported, including CommonJS and ECMAScript.\par
TypeScript was designed to address several shortcomings that have been present in the ecosystem for a lost time, especially when creating large-scale applications. JavaScript applications are very flexible with their dynamic and loosely typed nature and prototype usage, but with flexibility comes a large surface area for errors and mistakes. \par
Today, TypeScript is widely used for web development and JavaScript server-side development. Most popular frameworks provide at least partial support for TypeScript, and some (such as Angular and React) have even switched to it as the preferred language. TypeScript has support in many JavaScript-integrated development environments, such as Microsoft's Visual Studio Code or JetBrains WebStorm. With solid typing comes the ability for more substantial and consistent code completion, guaranteed automated refactoring, and error checking.\par
Other projects have tried to fix the same issues as TypeScript fixes. For example, Dart, which is developed by Google, works in the same way, although further from the traditional syntax, it is also compiled into standard JavaScript. However, it never gained the same traction, and its focus was changed from alternative to JavaScript to the primary language for development in the multi-platform framework Flutter.\par

\section{Node.js}
Node.js is an open-source, cross-platform JavaScript based on the V8 engine developed by Google for Google Chrome. It is designed to allow for server-side usage of JavaScript. Released by Ryan Dahl in 2009, it has since become standard for server-side JavaScript development, especially web applications. The framework has gained popularity thanks to its alternative execution model, which separates it from traditional server-side languages. Instead of spawning different threads or workers for connections, it works with a non-blocking asynchronous I/O model, where many concurrent connections can be handled with only a small overhead.\par
This is achieved through asynchronous programming, where multiple tasks can be executed concurrently without blocking the main execution. Node.js supports asynchronous programming through the concepts of callbacks and promises. Callbacks are functions passed as arguments that are executed in finished or failed states, ensuring that logic can be applied sequentially after the asynchronous operation is finished. Promises provide a more structured and object-focused way to handle asynchronous operations and have become the preferred way. A promise is a representation of value which might not be available yet, containing a status variable and reference for the result once achieved, allowing for code execution while the operation status is updated in the background. When the value of the promise is necessary, the promise can be checked or waited for using the async/await constructs.\par
While Node.js is currently the most dominant, there are other alternatives available with their own approaches and focuses. The most popular one is Deno, also developed by Ryan Dahl, intending to address some of the security and design issues of Node.js. Deno, for example, contains extensive tools and utilities within its standard library or uses better sandboxing between modules as supported by V8, the engine on which both it and Node.js run.\par

\section{npm}
One of the key benefits of the Node.js ecosystem is the large number of third-party packages that can be incorporated into projects. For example, many popular web frameworks, such as Koa or Express.js, are built for Node. Database drivers are also provided in module form, and therefore there needs to be a tool which allows users to incorporate such modules into their projects efficiently. \par
Originally an acronym for Node Package Manager, the three-letter name has been officially checked to the abbreviation of 'npm is not an acronym'. The first release was published in 2010, and it has since become the default Node.js package manager. Npm consists of a command line client, which is also called npm, and an online database of packages called the npm registry, which is hosted at \url{www.npmjs.com}. \par
Although npm is the default package manager, alternatives that were created with different focuses and compromises exist, for example, yarn.\par

\section{JSON}
JavaScript object notation (JSON) is a lightweight data-interchange format that is widely used in web development. Introduced as an alternative to the complex XML format that was previously used, it is based on a subset of JavaScript representation of values. It consists of key-value pairs in objects, arrays, and primitive types. One of the main benefits is its simplicity and readability for humans, which makes it useful for places where data could need to be interpreted by both humans and machines.\par
JSON has been standardised in the ECMA-4040 document by Ecma International. The document specifies syntax and semantics, ensuring its reliability, consistency, and portability throughout systems and applications.\par

\section{Unit of Work}
Unit of Work is a software design pattern used most commonly in ORMs and similar frameworks to manage persistence and consistency between application and database state. The pattern is used to group all database operations relating to a single transaction or process and only execute the final state, ensuring they can be performed atomically without requiring lengthy and expensive locking of database rows or tables or risking deadlocks through database transactions.\par
The main idea is to track changes across the object in memory, and instead of committing every change into the database, only the last state change is executed. This can be applied not only across one object instance but also across whole swathes of objects. While atomicity is undoubtedly necessary on many occasions, and unit of work on the ORM side can significantly reduce the number of requests to the database, it can also lead to inconsistency when multiple applications access the database and data which are currently loaded in memory on one machine are modified by a different one.\par

\section{Active record}
The Active Record pattern is a design pattern defined by Martin Fowler in his book "Patterns of Enterprise Application Architecture" and is commonly used to represent database records in an application.\par
The goal of the pattern is to encapsulate logic for interacting with the database table into a single object. Each instance of the object represents a single record, and modifications made on it are then usually flushed with a method call into the database. The base class also provides static methods for CRUD (create, read, update, delete) operations and possibly additional business logic.  \par

\textbf{INSERT EXAMPLE CODE FROM FOWLER}

The main benefit of the Active Record pattern is a simple and intuitive interface for objects and tables. Modifications of the object can be made right on the data in languages, which allow setters and getters on attributes, and static methods provide a simple gateway to work with the table. \par
Limitations of the pattern come in the tight coupling between the application and database logic, as the object instance is inherently tied to the database representation. This makes it harder to test the implementation and often requires additional abstraction or mocking. Additionally, the pattern does not easily allow for the management of relations, so a database schema with complex relations might not be able to represent the data easily. \par

\section{Data mapper}
The Data Mapper pattern, as described by Martin Fowler in his seminal work on enterprise application architectures, provides a clear separation between domain models and their underlying data storage. This approach enables developers to create complex and expressive domain models without being constrained by the relational database schema or various storage options. By decoupling in-memory representations from the data storage mechanisms, the Data Mapper pattern promotes a clean separation of concerns and enhanced flexibility in application design.\par
Distinguished from the Active Record pattern, the Data Mapper pattern ensures that business logic and data access responsibilities remain separate. In this approach, a single entity represents the table or collection, while distinct entities represent individual records. The Data Mapper serves as a data access layer that performs operations on the data storage representation without creating any direct bindings between in-memory objects and the database. This responsibility is solely managed by the Data Mapper, which takes care of any objects that utilise it.\par
This separation enables applications that employ the Data Mapper pattern to adhere to the Single Responsibility Principle, one of the SOLID principles of software design popularised by software engineer Robert C. Martin. By limiting the responsibilities class must service and ensure that it is not accountable for multiple unrelated tasks, the single responsibility principle aims to create more straightforward and more maintainable classes. Consequently, the Data Mapper pattern contributes to a more robust and modular software architecture that is easier to develop, maintain, and extend.\par
However, the Data Mapper pattern has drawbacks. One notable downside is the increased complexity introduced by the additional layer of abstraction. This added complexity could lead to a steeper learning curve for developers unfamiliar with the pattern, as well as the potential for increased development time. Moreover, the mapping process between domain objects and the persistence layer may introduce performance overhead, which can be a concern for applications with stringent performance requirements. Additionally, implementing the Data Mapper pattern often necessitates extensive configuration and mapping code, which can be time-consuming to write and prone to errors.\par

\section{MVC architecture}
The Model-View-Controller (MVC) architecture is a prevalent design pattern in software development, emphasising the separation of concerns by organising application components into three distinct roles. This architectural pattern, originating from the work of Trygve Reenskaug in the 1970s, has found widespread use in modern web development across various programming languages and frameworks.\par
The three components of the MVC architecture, Model, View, and Controller, each serve specific purposes. The model represents the application's underlying data structure and business logic, encapsulating core functionality, ensuring data consistency and handling the data storage and representation. In contrast, the view is tasked with rendering data and presenting them to users in an intelligible format. The controller functions as an intermediary between the model and the view, processing user input, manipulating the model, and updating the view as needed.\par
Separation of these components from the MVC architecture facilitates modularity, maintainability, and testability in software design. Each component can be developed, tested, and updated without interaction with the other layers, simplifying the development process and making it more manageable to identify and resolve issues. Furthermore, the separation of concerns allows developers to concentrate on a single aspect of the application at a time, resulting in more organised and efficient code.\par
However, the MVC architecture has drawbacks. One notable disadvantage is the added complexity resulting from the additional layers of abstraction, which might be challenging for inexperienced developers and could prolong the development process. Additionally, some critics contend that the strict separation of concerns can create a rigid structure that might need to be better suited for applications with rapidly changing requirements or unconventional designs. Additionally, the structure may be too complex for many projects, which would benefit from more concise and flexible architecture.
