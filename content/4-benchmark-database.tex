\chapter{Benchmark database schema design}\label{ch:database}

This chapter describes the database used for performance testing of the ORM and
query builder packages. The database is designed around imaginary data
collection about cats, their home domiciles and toys found within these houses,
and the toys' manufacturers. The database comprises six main entities -
\texttt{cat}, \texttt{cat colours}, colour \texttt{hex codes}, \texttt{houses},
\texttt{toys} and \texttt{toy producers}.

\begin{figure}[b]
    \caption{Schema of benchmark database}
    \centering
    \includegraphics[width=0.9\textwidth]{databaseSchema}
\end{figure}

\section{Cat Entity}
The \texttt{cat} entity instances represent individual cats which we want to
monitor. Each has a unique identifier, name and date of birth, all of which are
nullable except for the identifier. This entity aims to represent the basic
database table and to verify the correct handling of the data type from
Postgres, as JavaScript Date time represents a moment, including time. In
contrast, the database entry would only contain the date. Additionally, the
\texttt{cat} entity uses big integer data type, and handling numbers beyond the
standard range allocated in JavaScript is tested. The \texttt{cat colour} and
\texttt{colour hex code} are two entities that represent the cat colour by its
name and by its hex code. The entities are intentionally split in this way to
use identifying relation - the primary key of the hex colour entity is also a
foreign key referencing the id of the cat colour entity.

\section{House and Toy Entities}
The \texttt{house} entity represents domiciles where the cats spend their time
at their behest. The relation must also account for ambitious cats using several
houses as their homes. The main aim is to test the difficulty of implementing
and using simple many-to-many relations. The only attribute that provides new
data type or behaviour is the simple \verb|has_dog| attribute, specified as a
Boolean. It is one of several attributes that test the frameworks' ability to
correctly type and convert the data recovered from the database.

The houses can be equipped with many toys for the cats to use. The relation
between houses and toys is modelled through a decomposition table which contains
attributes representing the number of the same toy in the house. While the
primary keys are the identifiers of the house and toy, the decomposition with
the amount, rather than several records with an additional identifier, is
designed to test the ability to insert a record if it does not exist or update
the value referencing its previous state. If more toys are purchased, the owner
of the house does not suddenly throw out all toys they already had; they will
add them to their current pile. This operation is often called \textit{upsert} -
a combination of update and insert, and some database engines, such as
CockroachDB \cite{upsertCockroachDB}, implement it explicitly under this name.
PostgreSQL achieves it using the \texttt{ON CONFLICT} statement in
\texttt{INSERT} query \cite{INSERT_postgres_2023}. It also tests the handling of
composite primary keys, a standard paradigm in many databases.

\section{Toy Entity}
The \texttt{toy} entity purpose in testing is in numeric data type used in
\texttt{price} attribute and usage of additional column attributes such as
\texttt{CHECK} constraints or \texttt{DEFAULT} values in the column
\cite{Constraints_Postgres_2023}. Column \texttt{naughty} is focused on commonly
problematic strings in software development, such as special Unicode characters,
emojis and other issues that could come up in handling data from the database,
especially if the encoding is not correctly handled. Toys producers host the
JSON columns to test if it is possible to use advanced JSON traversal and query
operators provided in PostgreSQL \cite{postgres-json} (and their equivalents in
other database management systems).
