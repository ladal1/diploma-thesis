\section{Kysely}
Kysely is a~relatively young query builder in the TypeScript ecosystem, aiming
to replace Knex.js by offering the same query composition power while providing
superior type support for query composition and results. Although its
development only began in earnest in 2021, the package has gained traction since
2022, as it matured and continues to be actively developed.

Compared to Knex.js, Kysely supports fewer database engines and SQL dialects
out-of-the-box, with native support for MySQL, PostgreSQL, and SQLite. However,
third-party drivers are available for several other (albeit more exotic)
databases \cite{kyselyWeb}. One of the driving forces behind Kysely's creation
was Knex.js's excessive permissiveness, which limited its type support
capabilities \cite{kyselyGithub}. Kysely addresses this issue while retaining
the strengths of query composition. Though the order of operations in Kysely may
sometimes differ from SQL, this does not commonly pose a~significant problem.
The syntax is shown in \autoref{lst:kyselyExample}.

\begin{listing}[ht]
    \caption{Example of Kysely syntax}
    \label{lst:kyselyExample}
    \begin{minted}{typescript}
// Query for selecting toy id and the producer id, 
// for toys with price over 10000 
// and not produced by producer with id = 1

// Note the order of methods, 
// leftJoin has to come before select and where, 
// if values are selected or queried from the table joined
await kyselyInstance
    .selectFrom('toy')
    .leftJoin('toys_producer', 
        'toy.toys_producer_id', 
        'toys_producer.id')
    .select(['toy.id', 'toys_producer.id'])
    .where(({ or, not, cmpr }) =>
      or([
        not(cmpr('toys_producer.id', '=', 1)),
        cmpr('toy.price', '>', 10000),
      ])
    ).executeTakeFirstOrThrow()
\end{minted}
\end{listing}

Its excellent type guarantees make it an attractive choice, especially compared
to more expressive packages. Kysely can also be further enhanced with Model
classes through the third-party package kysely-orm. However, due to its limited
usage and lack of updates reflecting the latest Kysely API changes, it was not
considered for comparison here. To generate type definitions, developers can use
introspection with the \textit{kysely-codegen} package \cite{kyselyCodegen} or
an alternative schema specification via \textit{prisma-kysely}
\cite{kyselyPrisma}. Kysely also provides in built support for database
migrations, although it lacks the CLI that other packages, like Knex.js
provide.

Distributed with both CommonJS and EcmaScript dependency specifications, Kysely
is compatible with both dialects and is licensed under the MIT Licence. Despite
its youth, the package has already amassed over 53,000 weekly downloads on npm
and over 4,600 stars on GitHub \cite{kyselyGithub}. Most issues reported are
enhancement suggestions, however due to the package's young age, the relative
age and amount of issues is different.

Kysely's active development has led to frequent updates and enhancements, such
as improved documentation published during the writing of this thesis. While the
web document may be brief compared to more established packages like Knex, it
covers all essential information. The package's type annotations provide
excellent documentation, explanations, and multiple usage examples and are the
best example of function annotation in all packages considered in this
comparison. With zero dependencies apart from peer dependencies for database
drivers, Kysely is a~lightweight and efficient solution.

In summary, Kysely is a~promising query builder in the TypeScript ecosystem,
seeking to surpass Knex.js with its superior type support and powerful query
composition capabilities. Although it currently supports fewer database engines
and SQL dialects, its performance and type guarantees make it an attractive
choice for developers. Its active development, compatibility with various
dependency specifications, and high-quality documentation contribute to Kysely's
growing appeal within the TypeScript community.

\subsection*{Performance in benchmarks}

Regarding performance, the Kysely package consistently outperformed or at least
matched Knex, which is boasts a~comparatively rich set of features for a~query
builder. While Kysely does not offer parsing for types such as Big Integer, its
handling remains consistent with that of the pg driver. Like many other
packages, Kysely lacks support for explicit transaction control, above
encapsulation into one. However, this is at least supplemented with raw SQL
queries inside the encapsulation. This will however result in additional call of
\texttt{COMMIT} or \texttt{ROLLBACK} at the end of the encapsulation, even
though the transaction was finished manually, and such behaviour is usually
reserved for ORMs which abstract the database access further than common query
builders.

Another advantage of Kysely is its utilization of tagged strings, akin to the
approach employed by @databases/pg when composing raw SQL queries. This method
proves more comprehensible than the combination of bindings and function calls
implemented by Knex. Consequently, the Kysely package offers an appealing
balance of performance and usability, making it a~viable option for various
database management tasks.
