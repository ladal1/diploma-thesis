\section{TypeORM}

TypeORM is a pioneering ORM framework designed to support TypeScript and
leverage its features. It is compatible with both active directory and data
mapper patterns, allowing for easy exchange of entity definitions during
development. TypeORM supports a wide range of database engines, including MySQL,
PostgreSQL, SQLite, Oracle DB, and SAP Hana, and even has experimental support
for the NoSQL database MongoDB.

The framework employs its own query builder and uses Model classes with property
decorators to describe its methods. TypeORM supports Lazy Loading and Eager
loading when working with models and is compatible with CommonJS and ECMAScript.
It offers automatic migrations based on its models and manually written
migrations. However, it lacks support for seed files, requiring them to be
included as migrations if executed through its CLI.

TypeORM is highly popular, boasting the second-highest download number and the
highest amount of stars among those compared, with 1,192,427 weekly downloads
and 30,947 stars on GitHub. Released under the MIT License, TypeORM is the ORM
with the highest number of downloads and full support for TypeScript, as
Sequelize only outperforms it when considering the \textit{sequelize} package's
download numbers, not the \textit{sequelize-typescript} package used for comparison.

Despite its stable support, TypeORM's development has shifted towards
maintaining existing features in 2018 due to the amount of work for purely
volunteer supported development. This maintenance mode has led to 1,873 open
issues as of now. Although the project has now resumed development due to receiving
financial support, it primarily focuses on more minor updates every few weeks.
It is not as actively developed as other "active" projects in this analysis.

The documentation for TypeORM is of good quality, providing all the necessary
information for development. However, it needs more formatting and chapters,
making it easier to follow even if the developer does not know what precisely
they are looking for. The model methods include brief descriptions but are
mostly limited to a single-sentence short explanation.

TypeORM has the highest number of dependencies among the packages compared, with
many dependencies like "buffer" being necessary to support various runtimes,
including browser or React Native environments that lack the same standard
library support as Node.js. These dependencies are actively developed and have
high download numbers, posing no additional risk to the toolchain. However,
TypeORM's early adoption of the TypeScript ecosystem has largely overlooked its
JavaScript functionality, as evidenced by the documentation and example
projects.

\subsection*{Performance in benchmarks}

When examining the performance of TypeORM in various benchmarks, it was observed
that this ORM solution closely competes with its main rival, Sequelize. In most
tests, TypeORM either slightly outperformed or marginally fell behind Sequelize,
with negligible differences in performance. These outcomes can be attributed
minimally to disparities in the generated queries, as both ORMs are comparable
in their query generation and execution capabilities.

However, TypeORM's query generation approach raises some concerns when working
with complex joined tables. The generated queries are often complex for humans
to read due to excessive naming or using hashes instead of more recognizable
names. This complexity necessitates additional parsing for developers to
effectively understand and work with these queries, one such query is shown in
\autoref{lst:typeorm-query}. Moreover, TypeORM often prefers LEFT JOINs, which
are computationally more demanding than INNER JOINs and, in this case, will
yield the same output thanks to the database schema.

Another issue encountered with TypeORM relates to its handling of automatically
generated primary keys. The ORM disregards user input for these columns if they
can be generated automatically, limiting developers' flexibility when working
with databases. For instance, this constraint poses challenges when inserting
seed data with hard-coded IDs.

In terms of upsert functionality, TypeORM, like Sequelize, does not provide
support for increment operations. Additionally, it is impossible to formulate an
equivalent query using the included query builder, as the syntax does not allow
for custom update SQL and only supports ignoring specific data that would be
otherwise inserted.

When handling transactions, TypeORM lets developers create an explicit
transaction handler that includes entity manager methods and functions for
starting, managing, and ending transactions. While the filter types are only
lightly typed, they offer sufficient guidance for developers navigating the
available filtering options.

\begin{listing}
  \label{lst:typeorm-query}
  \caption{}
\begin{minted}{sql}
SELECT 
  "Toy"."id"               AS "Toy_id",
  "Toy"."toy_name"         AS "Toy_toy_name",
  "Toy"."barcode"          AS "Toy_barcode",
  "Toy"."price"            AS "Toy_price",
  "Toy"."currency"         AS "Toy_currency",
  "Toy"."naughty"          AS "Toy_naughty",
  "Toy"."date_introduced"  AS "Toy_date_introduced",
  "Toy"."toys_producer_id" AS "Toy_toys_producer_id"
FROM "toy" "Toy"
  LEFT JOIN "toy_house" "Toy__Toy_toyHouses" 
    ON "Toy__Toy_toyHouses"."toy_id" = "Toy"."id"
  LEFT JOIN "house" "Toy__Toy_toyHouses__Toy__Toy_toyHouses_house"
    ON "Toy__Toy_toyHouses__Toy__Toy_toyHouses_house"."id" = "Toy__Toy_toyHouses"."house_id"
  LEFT JOIN "house_cat" "859f7912a2d4ce14675c955dc00d5e1101503c58e113dba9c125d65aab7b94c"
    ON "859f7912a2d4ce14675c955dc00d5e1101503c58e113dba9c125d65aab7b94c"."house_id" =
      "Toy__Toy_toyHouses__Toy__Toy_toyHouses_house"."id"
  LEFT JOIN "cat" "521c7a07a3cd212cc1564d159554f82ebf641a8398676fa24cabb9988acfd85"
    ON "521c7a07a3cd212cc1564d159554f82ebf641a8398676fa24cabb9988acfd85"."id" =
      "859f7912a2d4ce14675c955dc00d5e1101503c58e113dba9c125d65aab7b94c"."cat_id"
WHERE ("521c7a07a3cd212cc1564d159554f82ebf641a8398676fa24cabb9988acfd85"."id" = 1)
\end{minted}
\end{listing}
