\section{@databases/pg}
The \texttt{@databases/pg} package is developed by Lindsay Forbes, a~prominent
contributor to the Node.js ecosystem with hundreds of published packages. The
package offers a~simple interface for CRUD operations on individual tables in
TypeScript. The package's typed interface is derived from schema definitions in
interfaces, which can be generated using the \texttt{@databases/pg-schema-cli}
\cite{pg-schema-cli} package. However, the package lacks support for table
joining, making it suitable only for basic queries. Complex queries require a
combination of templated SQL and conditions written using module functions with
types.

\begin{listing}
    \caption{countCatsByColor solution in @databases/pg showing where condition composition}
    \label{lst:atdatabases/example}
    \begin{minted}{typescript}
return db
  .query(
    sql`SELECT
      COUNT(*) as count
      FROM
        cat
      JOIN cat_color ON cat_color.id = cat.cat_color_id
      JOIN color_hex ON color_hex.id = cat_color.id
      WHERE ${dbTables
        .color_hex(db)
        .conditionToSql({ hex_code: hexCode }, 'color_hex')}`
      ).then(r => Number(r[0].count))
    \end{minted}
\end{listing}

Though not as feature-rich as a~typical ORM, the package simplifies basic
querying more than a~standard query builder. It is exported as an ES module and
was initially licensed under GPLv3 but is now published under the MIT License.
The \textit{@databases} project supports multiple databases via modular drivers,
with official support for PostgreSQL, MySQL, SQLite, and Expo/WebSQL
\cite{databases/pg}. Postgres has the most advanced support, with additional
databases supported through modular drivers.

The package has 517 stars on GitHub and 26,613 weekly downloads which makes it
the fourth least downloaded and the second least known on GitHub
\cite{databases/pg/npm}. The project is actively developed and comprises
numerous subpackages, with new databases being added and bugs frequently
addressed. The package depends solely on internal modules or widely used
packages, for example \texttt{assert-never} and \texttt{cuid}, along with the
\texttt{pg} driver as a~dependency.

While the documentation quality could be better, offering basic information, a
quick-start guide, and examples, there is no inline documentation requiring
developers to consult the online documentation. Database migrations can be
managed and executed using the \texttt{@databases/pg-migrations} package,
supporting both .sql and .ts formats. However, no interface is provided for
TypeScript migrations, necessitating custom scripts, and no schema modification
methods are included in the package.

@databases/pg is a~package that simplifies basic querying in TypeScript
by providing a~straightforward interface for CRUD operations. Although it lacks
the extensive functionality of a~traditional ORM, it offers compatibility with
multiple databases and has an active development community. However, the absence
of inline documentation and limited support for complex queries and migrations
may require developers to rely on additional resources and tools.

\subsection*{Performance in benchmarks}
As anticipated, packages with minimal abstraction, such as @databases/pg, tend
to exhibit superior performance in latency tests. Notably, this package is among
the few capable of automatically converting Big Integer values to their
corresponding JavaScript type.

The benchmarks depended a~lot on queries which were at least majorly manually
written as shown in \autoref{lst:atdatabases/example}, only providing shortcut
methods for basic operations, helpers for parameter insertion and parsers for
basic where binary comparisons.

Nevertheless, the \texttt{insertOrUpdate} method \cite{databases/pg}, meant for
upsert operations, has a~limitation, as it cannot differentiate between objects
designated for updates and those for inserts. This constraint poses challenges
for operations such as value incrementation or decrementation.

While @databases/pg does offer support for transactions, the package lacks an
API for managing these transactions. Instead, it only provides the functionality
to encapsulate operations within a~transactional context. This aspect should be
considered when evaluating the suitability of this package for particular
applications.

The package does not support integrated pagination, which is not unusual.
However, it also does not support \texttt{OFFSET} usage in its primary query
interface, which forced the usage of plain SQL for the composition of the query.
