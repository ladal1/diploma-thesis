\section{Objection.js}

Objection.js is an ORM based on the Knex.js query builder and is closely
connected to the Knex ecosystem. Developed partially by the same team, it
supports SQLite3, Postgres, and MySQL while also being compatible with many
other database engines supported by Knex. Unlike Sequelize or TypeORM,
Objection.js does not provide extensive abstraction; it only offers Models that
assist with query composition, types, and relation fetching. The Models
primarily serve as a starting point for SQL queries rather than offering the
complex functionality of ActiveDirectory or DataMapper patterns. 

The package has 128,872 weekly downloads and 6,972 stars on GitHub. Objection.js
predates the popularization of TypeScript. Its leading developer admits that the
package requires a significant rewrite of types to compete in TypeScript
support, which is not feasible due to the workload required. Objection.js was
not maintained for over a year, accumulating technical debt and leading to
incompatibility with TypeScript 4.8 and newer due to enhancements to the
\texttt{strictNullChecks} option. A new version has since been published, fixing
some issues that arose during that time. New developers have started working on
maintaining the project as of writing this thesis.

The package is released under the MIT Licence and offers high-quality web
documentation, including extensive usage guides and a detailed API Reference. A
unique feature of Objection.js is the ability to provide JSON Schema validation
for inserted objects, ensuring database data consistency.

Objection.js has only three dependencies for full functionality: Knex, AJV, and
db-errors. The AJV dependency provides JSON Schema validation, while db-errors,
originating from the same company that published Objection.js, aims to deliver a
unified API for handling various errors produced by different database engines.
Although the db-errors package has not been updated for over four years, it is
unlikely to pose a significant security threat to the package. Objection.js now
also supports both ESM and CommonJS, with the latest release fixing ESM
compatibility issues.

\subsection{Perfomance in benchmarks}

Objection.js was included in the performance benchmarks only after its typings
were updated to be compatible with TypeScript 4.8 and later while maintaining
\texttt{strictNullChecks}, a requirement for the proper functioning of other
packages like Zapatos. Once the compatibility issues were resolved, Objection.js
demonstrated performance largely on par with other ORM packages such as MikroORM
and TypeORM. However, a notable exception was observed in the getCatColor test,
where Objection.js was significantly slower than its counterparts.

This performance discrepancy was traced back to the method
\texttt{withGraphFetched}, the only type-supported way to fetch relations beyond
a single relationship in Objection.js without resorting to plain Knex query
builder calls. This method was selected for type support, but it also caused the
speed issue encountered. The method retrieves data through three separate,
dependent database calls, resulting in considerably longer query execution times
even with minimal latency. While this approach may offer some benefits in
specific scenarios, such as applying limits to individual queries to alleviate
the load on massive databases, it generally falls short of the performance
optimization capabilities offered by relational database management systems
(RDBMS) that utilize single, more complex queries.

Another challenge posed by Objection.js lies in its limited typing support. Many
inherited methods from Knex lack template variables to specify the result,
despite their original counterparts providing such functionality. Consequently,
developers often need to cast the results manually to ensure the correct final
type, except for the most basic queries.

The composition of Objection.js, which positions itself as not quite an ORM nor
a pure query builder, often requires developers to have direct access to the
query. However, this approach often obscures vital information, forcing
developers to rely on experience or testing to understand certain aspects of the
system. For instance, the table name is hidden when using predefined relations
for joining. Developers must determine the alias assigned to the joined table
for \texttt{WHERE} clauses without any clear guidance from the package.