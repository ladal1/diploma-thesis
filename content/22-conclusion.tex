\chapter{Conclusion}

In conclusion, this thesis has successfully investigated the spectrum of
database access options available to developers within the TypeScript ecosystem.
The research has led to the development of a~scalable benchmarking framework
designed to evaluate the capability of various packages to execute database
operations. This framework accommodates additional test databases and suites,
providing opportunities for future research and expansion.

Several avenues for future work emerge from the findings of this thesis. These
include extending the benchmark framework to incorporate a~more comprehensive
array of packages and databases, evaluating the capacity of these packages to
describe database schemas, and devising additional test suites to assess the
flexibility and performance of the packages under investigation.

The results of the benchmarking process provide valuable insights into the
performance and capabilities of several prominent ORM frameworks. While there
appeared to be negligible performance differences between TypeORM and Sequelize,
the two leading ORM frameworks, TypeORM demonstrated significantly better
support for TypeScript features, utilizing them to its advantage. Furthermore,
two lesser-known packages, Kysely and MikroORM, displayed remarkable utility and
flexibility in their respective domains.

Kysely, a~query builder alternative to the more popular Knex, exhibited superior
type support while delivering faster results and maintaining an equal capacity
for query creation. Kysely has significant potential to become a~compelling
alternative for developers seeking a~more type-safe and performant query
builder.

MikroORM surprised with its extensive feature set, excellent type support, and
the ability to leverage Knex for crafting highly complex queries. Its
performance in the benchmarks underscores the potential value of MikroORM for
developers who require a~more feature-rich and type-safe ORM solution while
still retaining the flexibility to utilize Knex when necessary.

The findings of this thesis emphasize the importance of considering lesser-known
packages in addition to the most popular solutions when selecting a~suitable ORM
framework or query builder. By doing so, developers can make more informed
decisions that align with their specific requirements, maximizing the benefits
of the chosen package and optimizing the overall development process.

As the ecosystem of TypeScript-compatible database access packages continues to
evolve, the framework can be updated and expanded to ensure that developers stay
well-informed about the strengths and weaknesses of emerging solutions. Such
service would enable them to make better-informed decisions when selecting a
package that best suits their project's needs, ultimately contributing to more
efficient and robust application development.

In summary, this thesis shed light on the diverse landscape of
TypeScript-compatible database access packages, providing insights into their
capabilities and performance. With the knowledge gained from this study,
developers stand better equipped to select the most suitable package for their
specific requirements, ensuring a~more streamlined and effective development
process.
