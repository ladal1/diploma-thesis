\chapter{Introduction}

In the modern world, data have become an essential aspect of almost every field.
From e-commerce to healthcare, education to finance, data is everywhere and
plays a critical role in decision-making processes. The advent of Web
2.0\cite{Web2Oreilly}, which brought with it the concept of user-generated
content, was largely supported by connecting the Web to databases. Social media
platforms, for example, rely heavily on data to provide personalized
recommendations, targeted advertising, and other features that keep users
engaged. Even non-Web entities and applications often need significant data
storage and, as a result, the ability to manage and manipulate data has become a
critical skill for developers and organisations alike.

Relational databases such as SQL Server and PostgreSQL are by far the most
popular type of databases for data storage used in business-level applications.
These databases use the relational data model, which is based on tables with
rows and columns, to store and manipulate data. However, there are also
non-relational NoSQL alternatives like MongoDB and Firestore that use a document
data model, key-value stores like Redis, or a graph data model (Neo4j) to manage
data. Although these databases have their unique strengths and weaknesses, they
are generally considered to be more flexible than relational databases and are
particularly well-suited for managing unstructured data.

Object-oriented programming languages and languages incorporating parts of the
paradigm, such as Java, Python, Ruby, and JavaScript, have gained
popularity\cite{stack-overflow-survey} due to their ability to create complex
software systems that can handle large amounts of data efficiently.
Object-oriented programming (OOP) is a programming paradigm that represents
concepts as \enquote{objects} that have attributes (data) and behaviours
(methods). This makes it easier to write, maintain, and reuse code, which is
essential when working with large-scale software systems.

Despite the popularity of object-oriented programming languages, there is often
a disparity between OOP languages and the relational data model used by many
databases. OOP languages are designed to work with objects, whereas relational
databases are designed to work with tables. This can make it challenging for
developers to work with databases using OOP languages. 

Object-Relational Mapping (ORM) has become a popular solution for developers who
need to connect object-oriented programming languages with relational databases
\cite{Torres_Galante_Pimenta_Martins_2017}. ORM allows developers to work with
relational databases using object-oriented programming languages, eliminating
the need to write complex SQL queries. By abstracting away the details of the
underlying database, ORM allows developers to focus on the application logic and
reduces the amount of boilerplate code that needs to be written. This makes it
easier for developers to work with databases and reduces the potential for
errors. 

The paper aims to conduct a comprehensive analysis of the most popular ORM
packages and SQL query builders for Typescript. This analysis will provide an
objective measurement of their relative strengths and weaknesses in terms of
functionality, type support, performance, and package quality. Also included are
noncomparative examples of syntax and usage to illustrate strengths and
weaknesses and to showcase the functionality of the modules. By evaluating each
package's performance in these key areas, the paper aims to provide a
comprehensive comparison that will be useful to developers who are looking for
the best ORM or SQL query builder package for their Typescript project.

Before we start with the full comparison of ORMs that support TypeScript, we
must first define what counts as an Object-Relational Mapping Package, what are
SQL Query Builders, and other technologies and terms used further in this work.
Then we explain how the packages further analysed were selected and by which
criteria they are ranked and reviewed.

