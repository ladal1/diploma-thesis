\section{pgTyped}
PgTyped, primarily developed by Adel Salakh \cite{pgTyped}, is an open-source
package that, while not technically an ORM or query builder, offers unique
functionality for TypeScript developers. It analyzes SQL queries written by
developers and, through database introspection, creates typed helper methods for
executing those queries. The final usage is shown in \autoref{lst:pgTyped}. This
package is designed specifically for TypeScript, resulting in excellent typing
but occasionally leading to complex errors, a common issue with other highly
typed language-based packages like Kysely.

The package is compatible only with PostgreSQL, as each database requires its
own parser and introspection process. PgTyped uses the 'pg' connection package,
which is only compatible with PostgreSQL. It provides complete SQL flexibility
with minimal overhead, allowing developers to use it without limitations that
come with pure SQL queries. However, it lacks the abstraction typically found in
ORM or query builder packages, making direct comparisons challenging.

PgTyped supports both ESM and CommonJS dependency initialization, ensuring
excellent compatibility at the expense of a slightly larger package size. It is
released under the MIT license, a highly permissive option. Its popularity ranks
relatively low, with its main package, "\texttt{@pgtyped/cli}" receiving only
about 10,000 downloads weekly \cite{pgtyped/cli}. On GitHub, it has more stars
but still falls within the lowest third of the packages compared.

\begin{listing}[ht]
  \caption{Usage of pgTyped}
  \label{lst:pgTyped}
  \begin{minted}{typescript}
  // -- File: EntityTraversal.sql
  /* @name countCatsByColor */
  SELECT
    COUNT(*)
  FROM
      cat
      JOIN cat_color ON cat_color.id = cat.cat_color_id
      JOIN color_hex ON color_hex.id = cat_color.id
  WHERE
      color_hex.hex_code = :hexCode;
  
  // -- File: EntityTraversal.ts 
  import {
    countCatsByColor,
  } from './EntityTraversal.queries'
  
  return countCatsByColor
    .run({ hexCode }, getClient())
    .then(result => Number(result[0].count))
  
  \end{minted}
\end{listing}

The package is well-supported, with regular version releases and active issue
resolution. Most long-lasting issues are feature requests rather than bug
reports. As the package is divided into development and runtime components, the
CLI package can afford more dependencies than combined packages. The runtime
dependency has only three direct dependencies: the widely-used \texttt{chalk}
and \texttt{debug} packages, and the parser dependency, which adds
\texttt{antlr4ts} for ANTLR 4 (ANother Tool for Language Recognition) grammar
functionality in TypeScript/JavaScript \cite{pgtyped/runtime}.

The documentation is brief \cite{pgtyped-docs}, providing examples and a quick
start guide that covers the essentials for using the package. While the
generated types lack method information, the original query is included in the
Javadoc annotation, making it easily accessible during development. PgTyped is a
unique and flexible package for TypeScript developers working with PostgreSQL.
However, it may provide a different abstraction level for those seeking a
traditional ORM or query builder solution.

\subsection*{Performance in benchmarks}
In line with expectations, the pgTyped package outperformed all other packages
examined in the study. Most pgTyped operations are executed before or during
compile time, yielding performance metrics comparable to those achieved using
the plain \textit{pg} driver. While the package does not offer explicit support for
parsing esoteric primitive types such as Decimal or Big Integer—resulting in
their return as strings (akin to the \textit{pg} driver's approach) — it does provide a
specific type for \texttt{JSON} and \texttt{JSONb} columns. However, these types
must be sufficiently generic, which limits their overall value.

As is typical with intricate TypeScript interfaces, the errors frequently
encountered when using pgTyped can be challenging to interpret and offer
limited guidance regarding type hinting. Despite this, the package proves
advantageous in error detection.

It is important to note that the pgTyped package solely supports
single-command queries. Consequently, each query had to be separated during the
transaction tests, with the commands for initiating and rolling back
transactions explicitly written as SQL code. This limitation warrants
consideration when selecting a package for implementation within a context.
