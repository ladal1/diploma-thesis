\subsection{RDB}

RDB, a lesser-known ORM package, was initially selected for comparison due to
its distinctive model definition approach and appearance in the npm repository
under the ORM tag. At the time of selection, RDB had only 284 weekly downloads
and 291 stars on GitHub, making it the least popular package among those
considered. What set RDB apart was its alternative model definition method,
which relied on method calls to create the entity model, unlike the schema-based
approaches employed by other frameworks.

However, further examination of RDB revealed several limitations that led to its
disqualification from the comparison. Firstly, the package does not provide any
types for methods beyond connection and model initialization. This lack of
typing support significantly hinders its compatibility with TypeScript, making
it challenging to work with in a typed environment.

Additionally, RDB does not offer a solution for circular dependency issues that
may arise when defining models across multiple files. Consequently, developers
must either use a singleton method to encapsulate the definition or repeat it
for each file, both of which are suboptimal approaches. Given these limitations,
RDB is most suited for single-file projects that can initially define their
database logic.

The framework also lacks typings for search method attributes and resulting
object types, further limiting its usefulness in TypeScript-based projects. Due
to these issues, as well as the file separation requirements of the benchmark,
RDB was ultimately disqualified from the comparison. The full interface is shown
in \autoref{lst:rdbModel} and as shown does not contain any methods for
referencing the model attributes.

\begin{listing}
\caption{RDB entity model type definition}
\label{lst:rdbModel}
\begin{minted}{typescript}
export interface Table {
    primaryColumn(column: string): ColumnDef;
    column(column: string): ColumnDef;
    join(table: Table): Join;
    hasMany(join: JoinRelation): HasMany;
    hasOne(join: JoinRelation): HasOne;
    formulaDiscriminators(...discriminators: string[]) : Table;
    columnDiscriminators(...discriminators: string[]) : Table;
}
\end{minted}
\end{listing}

In conclusion, while RDB presents an interesting alternative approach to model
definition, its limitations in typing support and handling of multi-file
projects make it less appealing for developers seeking a robust and flexible ORM
solution, mainly when working with TypeScript.