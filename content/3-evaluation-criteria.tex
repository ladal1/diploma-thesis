\chapter{Ranking and Grading of the Frameworks}

This chapter outlines and explains the criteria for evaluating ORM and SQL query
builder packages chosen for the comparison. These criteria will be the core
points which will be considered, but other specific notes will be made about
each package. The main criteria were the level of TypeScript support, range of
compatible database management systems, popularity, support, documentation
quality, dependency count, and performance in different scenarios.

\section{Quantifiable Criteria}

The main section of the evaluation criteria focuses on technical aspects of the
frameworks, specifically their usage of TypeScript, support for different
databases, and difficulty composing queries. As these qualities are
quantifiable, they were given the highest priority in comparing the packages.

\subsection{TypeScript Support}

The quality and extent of TypeScript support vary among the packages, with some
offering better integration and type safety without the need for casts. In
contrast, others only provide basic typing or require result type definitions to
be written into each request, which amounts to the same behaviour as if the
result was cast. Such functionality often comes when the package initially
written for JavaScript is not rewritten in TypeScript but is only provided with
a \texttt{types} file, which specifies call signatures, but cannot provide other
assurances.

\subsection{Database Compatibility}

Wide database compatibility is necessary when working with a large project that
may encompass many services or when choosing a toolchain for a team working with
dynamic technology stack, as the one database may not satisfy all the needs the
team might have, and building experience with multiple frameworks could be
considered unnecessary spending. Providing a unified API over multiple databases
can be one of the benefits of query builders or object-relational mapping
frameworks.

\subsection{Flexibility and Performance}

Flexibility and performance are crucial in a database access framework. Suppose
the package would restrict the ability to access the data, requiring roundabout
ways to deal with basic operations. In that case, there are better ways to
simplify development, just as if the framework creates excessively suboptimal
queries or adds excessive overhead. One of the requirements for a comprehensive
ORM framework is the ability to support many use cases and represent and work
with many different data models. If ORM doesn't support possible use cases or
cannot represent commonly used database design patterns, it is lacking in some
ways compared to one that does.

Performance is often secondary when choosing an ORM framework, as quite often,
even frameworks adding significant overhead and creating suboptimal queries are
usually not noticeably slowing down the application. As the application grows,
the performance can become significantly more critical, and the resources needed
can be more expensive to scale. A high-performing package can support this
growth by maintaining efficacy under load and effectively using its available
resources.

Performance was measured in multiple ways; the first metric was the execution
time of a single query to measure the latency added by using the framework,
compared to using other frameworks or plain database drivers. Benchmarking this
way provides information about the amount of overhead the framework requires to
function, and if the connection pool is well initialized, connections are
assigned optimally, and data are correctly retrieved. The second benchmark run
repeats the test multiple times to eliminate any inconsistency that could occur
in a single run.

\subsection{ECMAScript and CommonJS compatibility}
There are two different primary standards for JavaScript syntax, ECMAScript and
CommonJS. They primarily differ in how the inclusion of modules is written and
the mechanism of the module import. While CommonJS was dominant in the server
backend space for a long time, however, ECMAScript modules are becoming
significantly more popular, with support added in Node.js, TypeScript and many
popular packages.

Combining packages from both ecosystems can still lead to problems. The best way
to support all possible combinations is to provide both types of dependency
declarations.

\subsection{Licence}
As developers consider integrating packages into their projects, it is crucial
to consider and understand the significance of licences governing their use.
Open-source software is often regarded as a valuable resource, offering a large
amount of reusable code and often the best solution. However, it is essential to
understand that open-source does not necessarily equate to unregulated use.
Licences still dictate the terms under which the package and its code can be
employed, modified and redistributed. Therefore, developers must examine the
licences of potential packages to ensure their intended use aligns with the
terms granted.

The most permissive licences allow for usage, modification and redistribution
with at most a requirement to credit the original author/authors and don't
require the licensee to maintain the same licence in derivative works. Examples
of such permissive licences are \textit{MIT License} \cite{MITLicense} or
\textit{Apache License 2.0} \cite{ApacheLicense2}. These are generally preferable
for projects that demand flexibility in their use of the software.

While still free, the opposite side to the permissive licences are copyleft
licences, which impose more stringent requirements on the usage, especially
modifications and redistributions of the software. The primary example of such a
licence is \textit{GNU General Public License} \cite{GNUGPL}, which requires
derived works to be distributed under the same licence.

Since copyleft or other provisions might limit the usability of libraries such
as ORMs for many projects, it is necessary to include the licence as a grading
criterion.

\section{Package Properties Criteria}

However, technical criteria are only some that should be considered when
selecting a framework. Many of these factors are interconnected; often, success
in one is either caused by or preceded by doing well in others. For example,
while the popularity of the package can show the reliability and usability of
the package, it also often results in more issues being reported, and more
users are more likely to create community resources supplying or improving
official documentation.

\subsection{Popularity}

Popularity measures usage, as indicated by package downloads, the number of
issues, and the number of users on GitHub who have shown interest in the
repository. While all imperfect measures for absolute popularity, they help
compare popularity between packages by their relative difference. In the case
that the package usage requires multiple dependencies to be installed, for
example command line interface for development and runtime dependency, the
highest number is listed.

\subsection{Support}

The number of resolved and still open issues can be used to show popularity and
support of the project. With such a metric, support can be measured; however,
more important than that are the patterns of behaviour which maintainers have
shown previously. If the release schedule is predictable, bugs and security
issues are fixed quickly, hesitant adopters can be assured that this pattern
will continue, and the framework is a safe investment. On the contrary, a
project which is officially or probably no longer supported can be assumed to be
a wrong choice, as it cannot react to newly found errors and problems with
dependencies and might be unusable due to changes with TypeScript or Node.js
runtime.

\subsection{Dependencies}

As dependencies require maintenance due to their changes and vulnerable
versions, their amount should also be manageable. Otherwise, it might increase
the maintenance cost for the package and application size. Even though data
storage is less critical than previously, having a more storage-conscious
package is still beneficial.

\subsection{Documentation Quality}

Documentation quality is critical for new adoption and onboarding for working
with the framework. It also cannot be measured with reasonable objectivity.
Perceived quality depends on language understanding and users’ previous
experience with the programming language and similar frameworks. Evaluation of
documentation will therefore summarize clarity, extensiveness and whether
features such as JSdoc annotations \cite{Typescript-jsdoc} which can be parsed
by IDEs are used to contain or link to the documentation.


The following chapters aim to provide a comprehensive and in-depth analysis of
packages compared by evaluating each package by these comprehensive criteria
with additional added when.
